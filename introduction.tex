\chapter{Introduction}
Computer graphics studies the creation and manipulation of visual and
geometric content.  Subfields of study include image acquisition and
processing, the digital geometric modeling of objects, animation of these
models through time, and the rendering of models to produce visual
images.  Historically, animation for visual effects has focused largely on interpolation and
data-driven models.  While computationally less expensive, these
methods often produce results lacking realism.

In contrast, numerical simulation of phenomena based on mathematical
models has largely been the domain of computational physics and
mechanics.  With a focus on accuracy, many of the methods arising from
these fields of study are designed to be high order and to use a large
number of degrees of freedom. They are computationally demanding and
are often targeted to large supercomputers and other expensive
hardware.
  
As computing architectures have evolved, physics-based
simulation has become a tractable problem given the commodity
workstations and time constraints typical of the visual effects
industry.  As such new methods targeting these platforms have arisen from
the computer graphics community, providing new richness and realism to
computer animation.  Pioneering works include
\cite{terzopoulos:1987:elastic} for elastic deformable objects,
\cite{Foster:1996:RAO} for liquids, and \cite{rosenblum:1991:hair} for
hair.  

The introduction of physical simulation to visual effects
applications has given rise to new challenges. In contrast to most
engineering and computational physics applications, only physically
plausible simulations are needed.  As such, lower order models are
typically preferred with an emphasis on fast, visually compelling
effects over more accurate results.  Moreover, a level of artistic
directability is desired for most visual effects simulations. To this
end, artist-tunable material parameters as well as fast turnaround
times are needed, further emphasizing a need for efficient algorithms.
Today, with the increasing availability of chip multiprocessors and
graphics processing units (GPUs), there is a new emphasis on parallelizable
numerical schemes. Whereas some computer graphics problems, e.g.\ ray-tracing, are embarrassingly parallel, simulation typically
involves highly coupled elements which are difficult to break into
smaller independent problems.  Therefore, the development of
parallelizable simulation algorithms continues to be a challenging
problem.

The remainder of this work will focus on a number of simulation
techniques and numerical solvers targeted to visual effects.  In
Chapter~\ref{chap:poisson}, we will introduce a parallel
multigrid-preconditioned conjugate gradients algorithm for Poisson's
equation on voxelized domains. Applications to the simulation of
incompressible flows including smoke and free surface fluids will be
demonstrated.  In Chapter~\ref{chap:hair}, we will
present a hair collision algorithm which utilizes incompressible flow
techniques to precondition geometric hair collisions, allowing an
unprecedented number of hairs to be simulated.  Finally, in
Chapter~\ref{chap:elasticity}, we will introduce a new parallel multigrid
solver for corotational linear elasticity on voxelized domains, and we
will apply the solver to the problem of skeleton-driven,
collision-aware character skinning.  Much of the work presented here
has been previously published in papers for which I am the primary
author, namely \cite{mcadams:2009:hair,mcadams:2010:mgpcg,mcadams:2011:elasticity}.
