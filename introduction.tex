\chapter{Introduction}
Computer graphics studies the creation and manipulation of visual and
geometric content.  Subfields of study include image acquisition and
processing, the digital geometric modeling of objects, animation of these
models through time, and the rendering of models to produce visual
images.  Historically, animation for visual effects has focused largely on interpolation and
data-driven models.  While computationally less expensive, these
methods often produce results lacking realism.

In contrast, numerical simulation of phenomena based on mathematical
models has largely been the domain of computational physics and
mechanics.  With a focus on accuracy, many of the methods arising from
these fields of study are designed to be high order and to use a large
number of degrees of freedom. They are computationally demanding and
are often targeted to large supercomputers and other expensive
hardware.
  
As computing architectures have evolved, physics-based
simulation has become a tractable problem given the commodity
workstations and time constraints typical of the visual effects
industry.  As such new methods targeting these platforms have arisen from
the computer graphics community, providing new richness and realism to
computer animation.  Pioneering works include
\cite{terzopoulos:1987:elastic} for elastic deformable objects,
\cite{Foster:1996:RAO} for liquids, and \cite{rosenblum:1991:hair} for
hair.  

The introduction of physical simulation to visual effects
applications has given rise to new challenges. In contrast to most
engineering and computational physics applications, only physically
plausible simulations are needed.  As such, lower order models are
typically preferred with an emphasis on fast, visually compelling
effects over more accurate results.  Moreover, a level of artistic
directability is desired for most visual effects simulations. To this
end, artist-tunable material parameters as well as fast turnaround
times are needed, further emphasizing a need for efficient algorithms.
Today, with the increasing availability of chip multiprocessors and
graphics processing units (GPUs), there is a new emphasis on parallelizable
numerical schemes. Whereas some computer graphics problems, e.g.\ ray-tracing, are embarrassingly parallel, simulation typically
involves highly coupled elements which are difficult to break into
smaller independent problems.  Therefore, the development of
parallelizable simulation algorithms continues to be a challenging
problem.

The remainder of this work will focus on a number of simulation
techniques and numerical solvers targeted to visual effects.  In
Chapter~\ref{chap:poisson}, we will introduce a parallel
multigrid-preconditioned conjugate gradients algorithm for Poisson's
equation on voxelized domains. Applications to the simulation of
incompressible flows including smoke and free surface fluids will be
demonstrated.  In Chapter~\ref{chap:hair}, we will
present a hair collision algorithm which utilizes incompressible flow
techniques to precondition geometric hair collisions, allowing an
unprecedented number of hairs to be simulated.  Finally, in
Chapter~\ref{chap:elasticity}, we will introduce a new parallel multigrid
solver for corotational linear elasticity on voxelized domains, and we
will apply the solver to the problem of skeleton-driven,
collision-aware character skinning.  Much of the work presented here
has been previously published in papers for which I am the primary
author, namely \cite{mcadams:2009:hair,mcadams:2010:mgpcg}.
%% \section{Incompressible flows in computer animation}
%% Incompressible flows have proven very effective at generating many
%% compelling effects in computer graphics including free surface,
%% multiphase and viscoelastic fluids, smoke and fire.  
%% 	\subsection{Enforcing incompressibility via projection methods}
	
%% 	\subsection{Numerical discretizations}
%% 	One of the most popular techniques for enforcing the
%%         incompressibility constraint is MAC grid based projection as
%%         pioneered by \cite{HW65}. They showed that for simulations on
%%         a regular
%% Cartesian lattice, staggering velocity components on cell faces and pressures at cell centers gives rise to a natural means for projecting such a staggered velocity field to a discretely
%% divergence free counterpart. This projection is ultimately accomplished by solving a Poisson equation for the cell centered pressures. Categorization of the Cartesian domain into air,
%% fluid or solid cells gives rise to what \cite{BBB07} describe as the ``voxelized Poisson equation'' with unknown pressures and equations at each fluid cell, Dirichlet boundary conditions
%% in each air cell and Neumann boundary conditions at faces separating fluid and solid cells. This technique for enforcing incompressibility via voxelized Poisson was originally
%% popularized in computer graphics by \cite{Foster:1996:RAO} who used Successive Over Relaxation (SOR) to solve the system. \cite{FF01} later showed that incomplete Cholesky preconditioned conjugate
%% gradient (ICPCG) was more efficient. This has since become a very
%% widely adopted approach for voxelized Poisson and has been used for
%% smoke \cite{SRF05,MTPS04,FSJ01} and fire \cite{HG09} as well as
%% continuum models for solids such as sand \cite{zhu:2005:sand} and hair
%% \cite{MSWST09}. The voxelized pressure Poisson equation is also popular in free surface flows due to its efficiency and relative simplicity
%% \cite{CMT04,FF01,GBO04,HK05,HLYK08,KC07,MTPS04,NNSM08,WMT05,zhu:2005:sand}. 

%% There are alternatives to voxelized Poisson, e.g., \cite{BBB07} used a
%% variational approach that retains the same sparsity of discretization
%% while improving behavior on coarse grids for both smoke and free surface. Also, many non-MAC grid based approaches have been used to enforce incompressibility, typically for smoke rather than free surface simulations. For example, \cite{MCPTD09,KFCO06,FOK05,ETKSD07} use conforming tetrahedralizations to accurately enforce boundary conditions, \cite{LGF04} uses adaptive octree-based discretization, and \cite{CNF07} makes use of tetrahedralized volumes for free surface flow.

\section{Solids simulation for character animation}
	\subsection{Hair}
	
Simulating hair on a virtual character remains one of the most
challenging aspects of computer graphics. As hair is an integral
part of creating many virtual characters, the problem is especially important.
Unfortunately, the massive number of hairs interacting and colliding
makes this task especially challenging. Many approximations for
simulating hair exist, but they typically fail to provide the amount of
detail that real hair exhibits. Several applications, such as feature
films, aim to capture the high degree of complexity caused by several 
thousand interacting hair strands.

		\subsubsection{Strand dynamics}
		Many hair simulation techniques consider different geometric and constitutive models
for modeling individual hair strands or clumps of hair starting with
mass spring systems \cite{rosenblum:1991:hair} and projective dynamics
\cite{anjyo:1992:hair}.  Mass/spring simulated lattice deformers can
be added to model torsion
\cite{plante:2002:hair-complexity,selle:2008:hair}.
Rigid body chaining methods are also commonly used
\cite{brown:2004:real-time-knot,choe:2005:simulating-complex-hair,hadap:2006:orientedstrands}.
Techniques based on elastic rod theory have become popular
starting with \cite{pai:2002:strands} and continuing with
\cite{gregoire:2006:interactive-simulation-one-dimensional,bertails:2006:superhelices}.
\cite{spillmann:2007:corde} also uses elastic rod theory but
adaptively changes the number of points that define the hair curve.
Most recently, \cite{bergou:2008:elastic-rods} provides a novel
modeling of twist as deviation from a canonical frame as well as a
method to evolve this separately quasistatically. Work in strand
dynamics has been extensive, and our technique can be used with the
practitioner's model of choice, but for simplicity we use
the mass/spring model.

		\subsubsection{Hair interaction and complexity}
		
The simplest (but most expensive) approach to hair complexity
management is simulating and rendering every hair
\cite{rosenblum:1991:hair,anjyo:1992:hair,selle:2008:hair}. While this
theoretically yields the most detail, it is the most intractable,
especially if self-collisions are considered.  Thus most practitioners
use a clumping or continuum level of detail scheme that trades
accuracy for computational tractability:

\textbf{Continuum models.}  \cite{hadap:2001:continuum-hair} uses a Lagrangian
fluid (smoothed particle hydrodynamics) technique to model fluid forces on a
mass/spring system. While this efficiently handles bulk behavior, detailed hair
behavior is lost. In addition, since smoothed particle hydrodynamics (SPH) is
used, forcing fluid interaction through spring like forces, this approach is
effectively as numerically difficult as standard self-repulsion spring forces.
\cite{bando:2003:hair-loosely-connected-particles} also uses a continuum
approach while discarding particle connectivity for real-time
applications. Unfortunately, this results in visual artifacts in renderings
because of the lack of connectivity in addition to the continuum's loss of high
fidelity detail.  \cite{petrovic:2005:levelset-hair} presented a continuum
approach that rasterized hair to an Eulerian velocity field and level set with
the goal of reducing computational complexity, rather than improving on
fidelity. Dynamics on the grid was limited to repeated averaging (an
approximation of viscosity) to obtain friction effects. Unlike our method,
they did not consider the incompressibility or other fluid behaviors that could
be imparted on the velocity field. In addition, the goal of the work was to
model relatively simple hairstyles on stylized characters, so the detail lost by
using a coarse grid was acceptable for their application.

\textbf{Clumped models.}  There are many techniques that simulate hair
interactions through a sparse set of disjoint guides that collide with one
another
\cite{bertails:2006:superhelices,hadap:2006:orientedstrands,gupta:2006:real-time-hair,chang:2002:mutualinteractions}.
A greater set of rendered hair strands are then either interpolated between
these guides or clumped to the guide strands.
\cite{plante:2002:hair-complexity} used a lattice whose vertices were simulated
with a mass/spring model to define many more interpolated
hairs. \cite{choe:2005:simulating-complex-hair} synthesized additional hairs
from a sparse set of simulated guide hairs with a statistical
model. \cite{bertails:2006:superhelices} used a combination of interpolation and
clumping to generate rendered strands.  Many clumped techniques also use
adaptivity to merge or split groups of hair to reduce computation where and when
the required detail is low
\cite{bertails:2003:adaptive-wisp-tree,ward:2003:modeling-hair-lod,ward:2003:adaptive-grouping-hair}. The
use of sparse clumps often allows the use of cheap repulsion penalties as
opposed to geometric collisions because guides are expected to have
thickness, and in Section~\ref{sec:collisions} we will discuss how this breaks
down with thousands of guides.
	\subsection{Skeleton driven skinning}
	Creating appealing characters is essential for modern feature animation. One challenging aspect is the production of life-like deformations for soft tissues comprising both humans and animals. In order to provide the necessary control and performance for an animator, such deformations are typically computed using a skinning technique and/or an example based interpolation method. On the other hand, physical simulation of flesh-like material is usually avoided or relegated to an offline process due to its high computational cost. However, simulations create a range of very desirable effects, like squash-and-stretch and contact deformations. The latter is especially important as it can guarantee pinch-free geometry for subsequent simulations like cloth and hair. 
		\subsubsection{Kinematic methods}
		
Skeleton driven skin deformation was first introduced by \cite{Magnenat-Thalmann89}. Since then such techniques have been used extensively, especially the ``linear blend skinning'' technique (aka.\ ``skeleton subspace deformation'' (SSD) or ``enveloping''). However, the limitations of such techniques are well-known and have been the topic of numerous papers \cite{Wang02,Merry06,Kavan08}. Despite improvements, skinning remains, for the most part purely kinematic. It has proven very difficult to get more accurate, physically based deformations (e.g., from self-collisions and contact with other objects). Rather, such phenomena are typically created through a variety of example based approaches \cite{Lewis00,Sloan01}. Although very fast computationally, example based methods often require extreme amounts of user input, especially for contact and collision.
		\subsubsection{Physically based simulation}		
Simulation recently enabled significant advances to character realism in \cite{Irving:2008:SDF} and \cite{clutterbuck:2010:avatar}, albeit with the luxury of extreme computation time. Nevertheless these approaches demonstrated the promise of simulation. Many techniques reduce the accuracy of the elasticity model to help improve performance and interactivity. \cite{Waters90,Chadwick89} first demonstrated the effectiveness of comparatively simple mass/spring based approaches. \cite{Sueda:2008} add interesting anatomic detail using the tendons and bones in the hand, but use simple surface-based skin. \cite{Kry02} use principle component analysis of off-line elasticity simulation to provide interactive physically based SSD. \cite{capell:2005:pb,Capell:2002:ISD:566570.566622,Galopo07} used a skeleton based local rotational model of simple linear elasticity. \cite{Muller:2005:MDB} introduced shape matching, a technique that uses quadratic modal elements defined per lattice cell, allowing realtime albeit less accurate deformations. \cite{Rivers:2007:FFL} extended the accuracy of this
method while maintaining high performance with a fast SVD. Warped stiffness approaches \cite{Muller:2002:SRD,Muller:2004:IVM} are a more general example of the techniques developed by Cappel et al.\ and use an inexact force differential to yield easily solvable symmetric positive definite (SPD) linearizations. However, \cite{Chao:2010:SGM} recently demonstrated the importance of a more accurate approximation to rotational force differentials lacking in warped stiffness approaches. We  illustrate this important robustness limitation of the warped stiffness approximation in figure~\ref{fig:warpedstiff}. The instability of the method effectively precludes its use in skinning applications. Unfortunately, the more accurate linearizations  yield indefinite systems and thus require more expensive linear algebra techniques (e.g., GMRES). However we demonstrate a solution procedure in the present work that rivals the efficiency available to warped stiffness, but without the robustness difficulties.

