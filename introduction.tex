\chapter{Introduction}
Computer graphics studies the creation and manipulation of visual and
geometric content.  Subfields of study include image acquisition and
processing, the digital geometric modeling of objects, animation of these
models through time, and the rendering of models to produce visual
images.  Historically, animation for visual effects has focused largely on interpolation and
data-driven models.  While computationally less expensive, these
methods often produce results lacking realism.

In contrast, numerical simulation of phenomena based on mathematical
models has largely been the domain of computational physics and
mechanics.  With a focus on accuracy, many of the methods arising from
these fields of study are designed to be high order and to use a large
number of degrees of freedom. They are computationally demanding and
are often targeted to large supercomputers and other expensive
hardware.
  
As computing architectures have evolved, physics-based
simulation has become a tractable problem given the commodity
workstations and time constraints typical of the visual effects
industry.  As such new methods targeting these platforms have arisen from
the computer graphics community, providing new richness and realism to
computer animation.  Pioneering works include
\cite{terzopoulos:1987:elastic} for elastic deformable objects,
\cite{Foster:1996:RAO} for liquids, and \cite{rosenblum:1991:hair} for
hair.  

The introduction of physical simulation to visual effects
applications has given rise to new challenges. In contrast to most
engineering and computational physics applications, only physically
plausible simulations are needed.  As such, lower order models are
typically preferred with an emphasis on fast, visually compelling
effects over more accurate results.  Moreover, a level of artistic
directability is desired for most visual effects simulations. To this
end, artist-tunable material parameters as well as fast turnaround
times are needed, further emphasizing a need for efficient algorithms.
Today, with the increasing availability of chip multiprocessors and
graphics processing units (GPUs), there is a new emphasis on parallelizable
numerical schemes. Whereas some computer graphics problems, e.g.\ ray-tracing, are embarrassingly parallel, simulation typically
involves highly coupled elements which are difficult to break into
smaller independent problems.  Therefore, the development of
parallelizable simulation algorithms continues to be a challenging
problem.

The remainder of this work will focus on a number of simulation
techniques and numerical solvers targeted to visual effects.  In
Chapter~\ref{chap:poisson}, we will introduce a parallel
multigrid-preconditioned conjugate gradients algorithm for Poisson's
equation on voxelized domains. Applications to the simulation of
incompressible flows including smoke and free surface fluids will be
demonstrated.  In Chapter~\ref{chap:hair}, we will
present a hair collision algorithm which utilizes incompressible flow
techniques to precondition geometric hair collisions, allowing an
unprecedented number of hairs to be simulated.  Finally, in
Chapter~\ref{chap:elasticity}, we will introduce a new parallel multigrid
solver for corotational linear elasticity on voxelized domains, and we
will apply the solver to the problem of skeleton-driven,
collision-aware character skinning.  Much of the work presented here
has been previously published in papers for which I am the primary
author, namely \cite{mcadams:2009:hair,mcadams:2010:mgpcg}.


\section{Solids simulation for character animation}

	\subsection{Skeleton driven skinning}
	Creating appealing characters is essential for modern feature animation. One challenging aspect is the production of life-like deformations for soft tissues comprising both humans and animals. In order to provide the necessary control and performance for an animator, such deformations are typically computed using a skinning technique and/or an example based interpolation method. On the other hand, physical simulation of flesh-like material is usually avoided or relegated to an offline process due to its high computational cost. However, simulations create a range of very desirable effects, like squash-and-stretch and contact deformations. The latter is especially important as it can guarantee pinch-free geometry for subsequent simulations like cloth and hair. 
		\subsubsection{Kinematic methods}
		
Skeleton driven skin deformation was first introduced by \cite{Magnenat-Thalmann89}. Since then such techniques have been used extensively, especially the ``linear blend skinning'' technique (aka.\ ``skeleton subspace deformation'' (SSD) or ``enveloping''). However, the limitations of such techniques are well-known and have been the topic of numerous papers \cite{Wang02,Merry06,Kavan08}. Despite improvements, skinning remains, for the most part purely kinematic. It has proven very difficult to get more accurate, physically based deformations (e.g., from self-collisions and contact with other objects). Rather, such phenomena are typically created through a variety of example based approaches \cite{Lewis00,Sloan01}. Although very fast computationally, example based methods often require extreme amounts of user input, especially for contact and collision.
		\subsubsection{Physically based simulation}		
Simulation recently enabled significant advances to character realism in \cite{Irving:2008:SDF} and \cite{clutterbuck:2010:avatar}, albeit with the luxury of extreme computation time. Nevertheless these approaches demonstrated the promise of simulation. Many techniques reduce the accuracy of the elasticity model to help improve performance and interactivity. \cite{Waters90,Chadwick89} first demonstrated the effectiveness of comparatively simple mass/spring based approaches. \cite{Sueda:2008} add interesting anatomic detail using the tendons and bones in the hand, but use simple surface-based skin. \cite{Kry02} use principle component analysis of off-line elasticity simulation to provide interactive physically based SSD. \cite{capell:2005:pb,Capell:2002:ISD:566570.566622,Galopo07} used a skeleton based local rotational model of simple linear elasticity. \cite{Muller:2005:MDB} introduced shape matching, a technique that uses quadratic modal elements defined per lattice cell, allowing realtime albeit less accurate deformations. \cite{Rivers:2007:FFL} extended the accuracy of this
method while maintaining high performance with a fast SVD. Warped stiffness approaches \cite{Muller:2002:SRD,Muller:2004:IVM} are a more general example of the techniques developed by Cappel et al.\ and use an inexact force differential to yield easily solvable symmetric positive definite (SPD) linearizations. However, \cite{Chao:2010:SGM} recently demonstrated the importance of a more accurate approximation to rotational force differentials lacking in warped stiffness approaches. We  illustrate this important robustness limitation of the warped stiffness approximation in figure~\ref{fig:warpedstiff}. The instability of the method effectively precludes its use in skinning applications. Unfortunately, the more accurate linearizations  yield indefinite systems and thus require more expensive linear algebra techniques (e.g., GMRES). However we demonstrate a solution procedure in the present work that rivals the efficiency available to warped stiffness, but without the robustness difficulties.

