\title{Efficient solutions to voxelized discretizations of elliptic
problems with applications to physical simulation in visual effects}
\author{Aleka Anne McAdams}
\department{Mathematics}
\degreeyear{2011}

%%%%%% Committee %%%%%%
\chair {Joseph M.\ Teran}
\member{Christopher R.\ Anderson}
\member{Stanley J.\ Osher}
\member{Demetri Terzopoulos}

%%%%%% Acknowledgements %%%%%%
\acknowledgments{
First and foremost, I would like to thank Matthew Keegan for his invaluable support throughout graduate school.  He has helped me find my inner strength, for which I am forever grateful.  

I would also like to thank Alejandro Cantarero, Jeff Hellrung, Mike O'Brien, Hem Wadhar, and many others for their friendship throughout graduate school.  We have shared one of the best times of my life together.  I hope we can share many more.

Special thanks to Xiaoqin, my pet chinchilla, and the many other furry and feathery friends in my life.  Their unconditional love has brought me great joy.

I would like to thank Andy Selle and Eftychios Sifakis for their mentorship and collaboration, and later for their friendship. Along with Andy, I would like to thank Rasmus Tamstorf and Mark Empey for their collaboration and help finding my first job.

I would like to acknowledge my committee for their support.  Particularly, I would like to thank my advisor, Joey Teran, for his help and guidance in all of my work as well as opening many doors for me in industry.  Most of all, I am thankful for his belief in and support of my work.

The images in figures (\ref{fig:char}), (\ref{fig:collisions}), (\ref{fig:hand}), (\ref{fig:teaser}), (\ref{fig:thug}) were produced while working at Walt Disney Animation Studios and used for research purposes only.  They are copyrighted by Disney Enterprises and printed with permission.

Work appearing in chapter \ref{chap:poisson} was first published in \cite{mcadams:2010:mgpcg} with the following acknowledgement:
\begin{quote}
We wish to thank P. Dubey and the Intel Throughput Computing Lab for valuable feedback and support. Special thanks to Andrew Selle for his help with our audio-visual materials.  A.M., E.S. and J.T. were supported in part by DOE 09-LR-04-116741-BERA, NSF DMS-0652427, NSF CCF-0830554, ONR N000140310071. We wish to acknowledge the Stanford Graphics Laboratory and XYZrgb Inc. for the dragon model.
\end{quote}

The work presented in chapter \ref{chap:hair} was originally published in \cite{mcadams:2009:hair} with the acknowledgement:
\begin{quote}
A. McAdams is supported in part by NSF DMS-0502315. E. Sifakis and J. Teran are supported in part by DOE
09-LR-04-116741-BERA, NSF DMS-0652427, NSF CCF-0830554, ONR N000140310071, and
an Intel Larrabee Research Grant.  We would like to acknowledge Maryann Simmons,
Arthur Shek, and Joe Marks for useful discussions and support. Additionally, we
appreciate the Walt Disney Animation Studio artists providing us with our animated
character example.
\end{quote}

Finally, chapter \ref{chap:elasticity} will be published in \cite{mcadams:2011:elasticity} with the acknowledgement:
\begin{quote}
Y.Z. was affiliated with Walt Disney Animation Studios and UCLA while working on
this technique. We also thank David Kersey for his help with rendering. Part of this
research was supported by NSF DMS-0502315, NSF DMS-0652427, NSF CCF-0830554, DOE
09-LR-04-116741-BERA, ONR N000140310071, and ONR N000141010730.
\end{quote}
}

%%%%%% Vita %%%%%%
\vitaitem{1983} {Born, Anyang, South Korea.}
\vitaitem{2006} {B.A.~(Mathematics), DePauw University.}
\vitaitem{2006--2009}{Teaching Assistant, Mathematics Department, UCLA.}
\vitaitem{2007} {M.A.~(Mathematics), UCLA, Los Angeles, CA.}
%\vitaitem{2007--2009}{Teaching Assistant, Mathematics Department, UCLA.}
\vitaitem{2008} {Graduate Associate, Walt Disney Animation Studios.}
%\vitaitem{2008--2009}{Teaching Assistant, Mathematics Department, UCLA.}
\vitaitem{2009} {Academic Mentor, Research in Industrial Projects for Students (RIPS) at the Institute for Pure and Applied Mathematics, UCLA.}
\vitaitem{2009} {Research and Development Intern, Weta Digital.}
\vitaitem{2009--present} {Research Assistant, Mathematics Department, UCLA.}
\vitaitem{2010--2011} {Graduate Associate, Walt Disney Animation Studios.}
%\vitaitem{2010--present} {Research Assistant, Mathematics Department, UCLA.}

%%%%%% Publications %%%%%%
\publication {``Detail preserving continuum simulation of straight hair'', A.\ McAdams, A.\ Selle, K.\ Ward, E.\ Sifakis, and J.\ Teran, ACM Trans.\ Graph.\ (SIGGRAPH Proc.), 2009.}

\publication {``A parallel multigrid Poisson solver for fluids simulation on large grids'', A.\ McAdams, E.\ Sifakis, and J.\ Teran, Proc.\ of the 2010 ACM SIGGRAPH/Eurographics Symp.\ on Comp.\ Anim., 2010.}

\publication {``Efficient elasticity for character skinning with contact and collisions'', A.\ McAdams, Y.\ Zhu, A.\ Selle, M.\ Empey, R.\ Tamstorf, J.\ Teran, and E.\ Sifakis, ACM Trans.\ Graph.\ (SIGGRAPH Proc.), 2011. \emph{to appear.}}


%%%%%% Abstract %%%%%%
\abstract{The introduction of physical simulation to visual effects
applications has given rise to new challenges.  In contrast to most
engineering and computational physics applications, lower order models are
typically preferred with an emphasis on fast, visually compelling
effects over more accurate results. To this
end, artist-tunable material parameters as well as fast turnaround
times are needed, further emphasizing a need for efficient
algorithms. We present a number of simulation
techniques and numerical solvers targeted to visual effects. In
particular, we
introduce a parallel
multigrid-preconditioned conjugate gradients algorithm for Poisson's
equation on voxelized domains and demonstrate applications to the simulation of
incompressible flows including smoke and free surface fluids; we
present a hair collision algorithm which utilizes incompressible flow
techniques to precondition geometric hair collisions, allowing an
unprecedented number of hairs to be simulated; and finally, we introduce a new parallel multigrid
solver for corotational linear elasticity on voxelized domains and apply the solver to the problem of skeleton-driven,
collision-aware character skinning.}
