Symmetry of the stiffness matrix $\mathbf{K}$ allows the use of certain Krylov
methods, but positive definiteness is required for conjugate gradients. While
$\mathbf{K}$ will be positive definite close to equilibrium (it is the energy
Hessian), in practice Newton-Raphson may generate intermediate
indefinite states. Like \cite{teran:2005:quasistatics}, we will modify
$\mathbf{K}$ to guarantee definiteness while retaining the same nonlinear
solution, a valid equilibrium configuration (though maybe via different iterates).
% \RT{Is this true ? The Newton scheme  will still converge, but will it converge to the same solution ?}
%This
%modified Newton scheme does not change the computed nonlinear solution; the
%algorithm converges to a perfectly valid equilibrium configuration, but while
%iterating through a different sequence of estimates than the original
%Newton-Raphson method.


%in order to use the most efficient variants such as
%conjugate gradients we additionaly require $\mathbf{K}$ to be positive definite. Since  is also the energy Hessian, it will be positive definite in the immediate vicinity of an
%equilibrium configuration. In practice however, for skinning simulations where large skeletal motions between frames are common, the Newton-Raphson procedure will very frequently
%generate indefinite systems before a converged solution is reached. In the spirit of \cite{teran:2005:quasistatics} we will enable the robust use of CG in this context, by modifying
%the stiffness matrix $\mathbf{K}$ so that its definiteness is guaranteed. 

Similar to \cite{teran:2005:quasistatics} we will conservatively enforce the definiteness of $\mathbf{K}$ by projecting each elemental stiffness matrix to its positive semi-definite
counterpart, i.e.\ a matrix with the same eigenvectors, but with negative eigenvalues clamped to zero. Naturally, we want to avoid an explicit eigenanalysis of the $24\times 24$
elemental stiffness $\mathbf{K}^e$, and even avoid forming it at altogether. We describe a procedure to perform this semi-definite projection in an inexpensive, matrix-free fashion. We
will initially describe the definiteness projection for the simple one-point quadrature rule, defined in section \ref{sec:elasticity}. This will be a stepping stone in designing a
definiteness fix for our stabilized quadrature rule, described in
section \ref{sec:stabilization}. 

The elemental stiffness matrix is positive semi-definite if and only if
$0\leq\delta\vec{x}^T\mathbf{K}^e\delta\vec{x}=-\delta\vec{x}^T\delta\vec{f}$, where $\delta\vec{x}$, $\delta\vec{f}$ are the stacked nodal position and force differentials for
$\Omega_e$. Taking differentials on both sides of equation (\ref{eqn_nodal_forces}) we get: 
$$
\delta f_i^{(j)}
=-V_e\sum_k\delta P_{jk}G_{ki}^e
=-V_e\sum_{k,l,m}T_{jklm}\delta F^e_{lm}G_{ki}^e
$$
where $\mathcal{T}=[T_{ijkl}]$ is the fourth order tensor defined as the stress derivative $\mathcal{T}:=\partial\mathbf{P}/\partial\mathbf{F}$, or $T_{ijkl}=\partial P_{ij}/\partial
F_{kl}$. We then write:
\begin{align*}
-\delta\vec{x}^T\delta\vec{f}&=-\sum_{i,j}\delta f_i^{(j)}\delta x_i^{(j)}\\
&=V_e\!\!\sum_{j,k,l,m}\!\!T_{jklm}\delta F^e_{lm}\sum_iG_{ki}^e\delta
x_i^{(j)}\\
&=V_e\!\!\sum_{j,k,l,m}\!\!T_{jklm}\delta F^e_{lm}\delta F^e_{jk}
\end{align*}
Thus, $\mathbf{K}^e$ will be positive semi-definite, if and only if the fourth order tensor $\partial\mathbf{P}/\partial\mathbf{F}$ is positive definite as well (in the sense that
$\delta\mathbf{F}:\mathcal{T}:\delta\mathbf{F}\geq 0$, for all
$\delta\mathbf{F}$). 

At this point, consider a different 4th order tensor $\hat{\mathcal{T}}$ defined by
$\delta\mathbf{P}=\mathcal{T}:\delta\mathbf{F}=\mathbf{R}[\hat{\mathcal{T}}:(\mathbf{R}^T\delta\mathbf{F})]$. Intuitively, if we define the \emph{unrotated} differentials
$\delta\hat{\mathbf{P}}=\mathbf{R}^T\delta\mathbf{P}$, and $\delta\hat{\mathbf{F}}=\mathbf{R}^T\delta\mathbf{F}$, then $\hat{\mathcal{T}}$ is the tensor that maps
$\delta\hat{\mathbf{P}}=\hat{\mathcal{T}}:\delta\hat{\mathbf{F}}$. Tensors $\mathcal{T}$ and $\hat{\mathcal{T}}$ are a similarity transform of one another; consequently they share the
same eigenvalues, and performing the indefiniteness fix on one will guarantee the definiteness of the other. Using this definition and equations (\ref{eqn_stress_differential}) and
(\ref{eqn_rotation_differential}), $\delta\hat{\mathbf{P}}$ reduces to:
\begin{equation}
\delta\hat{\mathbf{P}}=\hat{\mathcal{T}}:\delta\hat{\mathbf{F}}=
2\mu\delta\hat{\mathbf{F}}\!+\!\lambda\tr(\delta\hat{\mathbf{F}})\mathbf{I}\!+\!\left\{\lambda\tr(\mathbf{S}\!-\!\mathbf{I})\!-\!2\mu\right\}\mathcal{S}\!:\!\delta\hat{\mathbf{F}}
\label{eqn_rotated_dp}
\end{equation}
where $\mathcal{S}=\mathcal{E}:\left\{\tr(\mathbf{S})\mathbf{I}-\mathbf{S}\right\}^{-1}:\mathcal{E}^T$. Consider the decomposition of
$\delta\hat{\mathbf{F}}\!=\!\delta\hat{\mathbf{F}}_{\mbox{\small sym}}\!+\!\delta\hat{\mathbf{F}}_{\mbox{\small skew}}$ into symmetric 
$\delta\hat{\mathbf{F}}_{\mbox{\small sym}}\!=\!(\delta\hat{\mathbf{F}}\!+\!\delta\hat{\mathbf{F}}^T)/2$ and skew symmetric
$\delta\hat{\mathbf{F}}_{\mbox{\small skew}}\!=\!(\delta\hat{\mathbf{F}}\!-\!\delta\hat{\mathbf{F}}^T)/2$ parts; also consider a similar decomposition of 
$\delta\hat{\mathbf{P}}\!=\!\delta\hat{\mathbf{P}}_{\mbox{\small sym}}\!+\!\delta\hat{\mathbf{P}}_{\mbox{\small skew}}$. By collecting symmetric and skew symmetric terms from equation
(\ref{eqn_rotated_dp}) we have:

\begin{equation}
\delta\hat{\mathbf{P}}_{\mbox{\small sym}}=2\mu\delta\hat{\mathbf{F}}_{\mbox{\small sym}}\!+\!\lambda\tr(\delta\hat{\mathbf{F}}_{\mbox{\small sym}})\mathbf{I}=\mathcal{T}_{\mbox{\small sym}}:\delta\hat{\mathbf{F}}_{\mbox{\small sym}}
\label{eqn_dp_sym}
\end{equation}
\begin{equation}
\hspace*{-.2in}\delta\hat{\mathbf{P}}_{\mbox{\small skew}}
\!=\!2\mu\delta\hat{\mathbf{F}}_{\mbox{\small skew}}\!+\!\left\{\lambda\tr(\mathbf{S}\mbox{-}\mathbf{I})\mbox{-}2\mu\right\}\mathcal{S}\!:\!\delta\hat{\mathbf{F}}_{\mbox{\small skew}}
\!=\!\mathcal{T}_{\mbox{\small skew}}\!:\!\delta\hat{\mathbf{F}}_{\mbox{\small skew}}
\label{eqn_dp_skew}
\end{equation}

In essence, $\hat{\mathcal{T}}=\mathcal{T}_{\mbox{\small sym}}+\mathcal{T}_{\mbox{\small skew}}$ has a fully decoupled action on the two subspaces of symmetric, and skew symmetric
matrices. Since the symmetric and skew subspaces are orthogonal, $\hat{\mathcal{T}}$ will be semi-definite, if and only if its skew and symmetric parts are semi-definite too. The
tensor $\mathcal{T}_{\mbox{\small sym}}=2\mu\mathcal{I}_{\mbox{\small sym}}+\lambda\mathbf{I}\otimes\mathbf{I}$ ($\mathcal{I}_{\mbox{\small sym}}$ is the operator that projects a
matrix onto its symmetric part) is always positive semi-definite; thus no modification is necessary. If $\mathcal{I}_{\mbox{\small skew}}$ is the operator that projects a matrix onto
its skew symmetric part, we can verify that $2\mathcal{I}_{\mbox{\small skew}}=\mathcal{E}:\mathbf{I}:\mathcal{E}^T$. Thus, $\mathcal{T}_{\mbox{\small skew}}$ is written as:
$$
\mathcal{T}_{\mbox{\small skew}}=\mu\mathcal{E}:\mathbf{I}:\mathcal{E}^T+\left\{\lambda\tr(\mathbf{S}\!-\!\mathbf{I})\!-\!2\mu\right\}\left[\mathcal{E}\!:\!\left\{\tr(\mathbf{S})\mathbf{I}\!-\!\mathbf{S}\right\}^{-1}\!:\!\mathcal{E}^T\right]
$$
$$
=\mathcal{E}\!:\!\mathbf{L}\!:\!\mathcal{E}^T,\ \mbox{where}\ \mathbf{L}=\mu\mathbf{I}+\left\{\lambda\tr(\mathbf{S}\!-\!\mathbf{I})\!-\!2\mu\right\}\left\{\tr(\mathbf{S})\mathbf{I}\!-\!\mathbf{S}\right\}^{-1}
$$
$\mathcal{E}$ is also an orthogonal (although not orthonormal) tensor, thus the definiteness of $\mathcal{T}_{\mbox{\small skew}}$ is equivalent with the definiteness of the $3\times 3$ 
symmetric matrix $\mathbf{L}$, which can be easily projected to its positive definite part. In fact, if the method we used to compute the polar decomposition were to first compute the
entire SVD $\mathbf{F}=\mathbf{U}\mathbf{\Sigma}\mathbf{V}^T$ of the deformation gradient, then we have $\mathbf{L}=\mathbf{V}\mathbf{L}_D\mathbf{V}^T$ where
$$
\mathbf{L}_D=\mu\mathbf{I}+\left\{\lambda\tr(\mathbf{\Sigma}\!-\!\mathbf{I})\!-\!2\mu\right\}\left\{\tr(\mathbf{\Sigma})\mathbf{I}\!-\!\mathbf{\Sigma}\right\}^{-1}
$$
is a diagonal matrix, whose diagonal matrices simply need to be clamped to zero, to ensure definiteness for $\mathbf{L}$, for $\mathcal{T}_{\mbox{\small skew}}$ and ultimately for the entire
element stiffness matrix. In practical implementation, the matrix $\mathbf{L}$, projected to its semi-definite component, is precomputed and stored at the same time when the Polar Decomposition
of each element is performed. Then, the definitions of this section are followed to successively construct 
$\delta\hat{\mathbf{P}}_{\mbox{\small sym}}$ and \ $\delta\hat{\mathbf{P}}_{\mbox{\small skew}}$, using equations (\ref{eqn_dp_sym}),(\ref{eqn_dp_skew}), and ultimately
$\delta\mathbf{P}=\mathbf{R}\delta\hat{\mathbf{P}}$. 

So far, we have discussed how to correct the indefiniteness of the stiffness matrix arising from the (unstable) one-point quadrature technique. In light of the energy decomposition reflected
in equation (\ref{eqn_psi_poisson_and_aux}) the difference in the discrete energy between the stable and unstable approaches, is the discrete quadrature that will be followed to
integrate the part $\Psi_\Delta$. Our stable technique employs equation (\ref{eqn_quadrature_poisson}) for this task, while the original unstable technique uses one-point
quadrature. In two spatial dimensions, if we denote by $E_\Delta^S$ and $E_\Delta^U$ the discrete integral associated with the Laplace term in the stable, and unstable variants
respectively, we then have:
\begin{eqnarray*}
E_\Delta^U&=&\mu h^2\sum_{i=1}^2\left((F_{i1}^e)^2+(F_{i2}^e)^2\right)\\
&=&\mu h^2\sum_{i=1}^2\left((\frac{F_{i1}^A+F_{i1}^B}{2})^2+(\frac{F_{i2}^C+F_{i2}^D}{2})^2\right)
\end{eqnarray*}
$$
E_\Delta^U-E_\Delta^S\stackrel{\mbox{\small(\ref{eqn_quadrature_poisson})}}{=}\mu h^2\sum_{i=1}^2\left((\frac{F_{i1}^A-F_{i1}^B}{2})^2+(\frac{F_{i2}^C-F_{i2}^D}{2})^2\right)\geq 0
$$
Thus, we can interpret our stable discretization as adding the unconditionally convex term $E_\Delta^U-E_\Delta^S$ to the unstable energy discretization of the single-point
approach. The indefiniteness fix described in the context of the unstable method can also be interpreted as augmenting the real stiffness matrix with a supplemental term
$\mathbf{K}\gets\mathbf{K}+\mathbf{K}_{\mbox{\small supp}}$ that guarantees the definiteness of the resulting matrix. The last equation indicates that if the same
``definiteness-boosting'' matrix $\mathbf{K}_{\mbox{\small supp}}$ is added to the stable discretization, definiteness will be guaranteed. Algorithm \ref{alg_indefiniteness_fix}
summarizes the entire procedure that implements the ``auxiliary'' stress differential corresponding to the $\Psi_{\mbox{\small aux}}$ energy component. The differential of the
additional force due to the Laplace term $\Psi_\Delta$ are computed as described in section \ref{sec:stabilization}.
\begin{algorithm}[h]
\caption{Computation of the stress differential corresponding to the auxiliary energy term $\Psi_{\mbox{\small aux}}$. Fixed to guarantee definiteness.}
\label{alg_indefiniteness_fix}
\begin{algorithmic}[1]
\Function{Compute\_L}{$\mathbf{\Sigma},\mathbf{V},\mu,\lambda,\mathbf{L}$}
\State $\mathbf{L}_D\gets\left\{\lambda\tr(\mathbf{\Sigma}\!-\!\mathbf{I})\!-\!2\mu\right\}\left\{\tr(\mathbf{\Sigma})\mathbf{I}\!-\!\mathbf{\Sigma}\right\}^{-1}$
\State Clamp diagonal elements of $\mathbf{L}_D$ to a minimum value \\\nonumber\hspace*{.25in}of $(-\mu)$\Comment{Term $\Psi_\Delta$ will boost this eigenvalue by $\mu$}
\State $\mathbf{L}\gets\mathbf{V}\mathbf{L}_D\mathbf{V}^T$
\EndFunction
\Function{dPauxDefiniteFix}{$\delta\mathbf{F},\mathbf{R},\mathbf{L}$}\Comment{Returns $\delta \mathbf{P}_{\mbox{\small aux}}$}
\State $\delta\hat{\mathbf{F}}_{\mbox{\small sym}}\gets\Call{SymmetricPart}{\mathbf{R}^T\delta\mathbf{F}}$
\State $\delta\hat{\mathbf{F}}_{\mbox{\small skew}}\gets\Call{SkewSymmetricPart}{\mathbf{R}^T\delta\mathbf{F}}$
\State $\delta\hat{\mathbf{P}}_{\mbox{\small sym}}\gets\lambda\tr(\delta\hat{\mathbf{F}}_{\mbox{\small sym}})\mathbf{I}$
\State $\delta\hat{\mathbf{P}}_{\mbox{\small skew}}\gets\mathcal{E}\!:\!\left\{\mathbf{L}(\mathcal{E}^T\!:\!\delta\hat{\mathbf{P}}_{\mbox{\small skew}})\right\}$
\State $\delta\mathbf{P}_{\mbox{\small aux}}\gets\mathbf{R}\left(\delta\hat{\mathbf{P}}_{\mbox{\small sym}}+\delta\hat{\mathbf{P}}_{\mbox{\small skew}}\right)$
\State \Return $\delta\mathbf{P}_{\mbox{\small aux}}$
\EndFunction
\end{algorithmic}
\end{algorithm}



\section{Dynamics}
Our method extends trivially to dynamic simulations that include inertial effects. However, it is important to note that the indefiniteness encountered in quasistatic time stepping also arises in implicit time stepping for dynamics. Fortunately, the definiteness fix outlined above can be used in this setting as well. Here we will outline why this indefiniteness is increasingly likely to occur when performing interactive high-resolution simulation. In this case, we typically desire a fixed temporal resolution of $\Delta{t}\approx\frac{1}{30}$ (to take as few timesteps as possible). 

Using the common backward Euler time stepping scheme for illustration, and assuming that the resulting non-linear equations are solved with Newton-Raphson's method, the following update equation must be solved for the increment $\delta\vec{{x}}$ in the $k^\textrm{th}$ iteration:
\begin{align}
\mathbf{K}^{BE}(\mathbf{x}^{n+1}_k)\delta\mathbf{x} &= \mathbf{f}(\mathbf{x}^{n+1}_k) + \mathbf{g}+\mathbf{M}\left(\left(\mathbf{x}^n-\mathbf{x}^{n+1}_k\right)/\Delta t^2 + \mathbf{v}^n/\Delta t\right)\nonumber\\
\mathbf{x}^{n+1}_{k+1}&=\mathbf{x}^{n+1}_k+\delta\mathbf{x}\label{eqn:backward_euler}
\end{align}
% To demonstrate this fact, let us examine the comparatively simple case of backward Euler time stepping. Here, our discrete system of equations is:
% $$
% \frac{\vec{x}^{n+1}-\vec{x}^{n}}{\Delta{t}} = \vec{v}^{n+1}\ \ \mbox{and}\ \ 
% \mathbf{M}\left(\frac{\vec{v}^{n+1}-\vec{v}^{n}}{\Delta{t}}\right) = \vec{f}(\vec{x}^{n+1})
% $$
% where $\mathbf{M}$ is the mass matrix associated with the material of interest. 
% %
% % For example, in a finite element based discretization one would use $M_{ij}=\int_{\Omega_0}{\rho(\mathbf{X})}{N_i}{N_j}d\mathbf{X}$ where $N_i$ and $N_j$ are the interpolating functions associated with each degree of freedom and $\rho$ is the mass density of the material. However, in graphics it is common to use a diagonal approximation (or mass lumping approximation) to this matrix obtained by summing each row: $M^\textrm{d}_{ii}=\sum_jM_{ij}$. This is equivalent to assigning each node an equal portion of volume of each of its incident elements and then determining the mass of the node as density times this volume. For example, in a tetrahedron based approximation each node would be given one fourth of the volume of each of its incident tetrahedra. 
% %
% We can solve this non-linear system by first eliminating $\vec{v}^{n+1}$ to get
% $$
% \mathbf{M}\left(\vec{x}^{n+1}-\vec{x}^{n}\right) = \Delta{t}\mathbf{M}\vec{v}^n + \Delta{t}^2\vec{f}(\vec{x}^{n+1})
% %\mathbf{M}\left(\frac{\frac{\vec{x}^{n+1}-\vec{x}^{n}}{\Delta{t}}-\vec{v}^{n}}{\Delta{t}}\right)=\vec{f}(\vec{x}^{n+1})
% $$
%% and then performing Newton iteration for $\vec{x}^{n+1}$. At the $k^\textrm{th}$ Newton iteration, we update an approximation to $\vec{x}^{n+1}$ given by $\vec{x}^{n+1}_{k+1}= \vec{x}^{n+1}_{k}+ \vec{\delta{x}}$ by solving the system
%$$
%\mathbf{K}^\textrm{BE}(\vec{x}^{n+1}_{k}) \delta\vec{{x}}=\Delta{t}\mathbf{M} \vec{v}^n+\Delta{t}^2 \vec{f}(\vec{x}^{n+1}_{k})
%$$
Here $\mathbf{K}^{BE}(\mathbf{x}^{n+1}_k) = \mathbf{M}/\Delta{t}^2+\mathbf{K}(\vec{x}^{n+1}_{k})$, and $\mathbf{M}$ is the mass matrix. 

The indefiniteness of $\mathbf{K}(\vec{x}^{n+1}_{k})$ can thus be seen to potentially cause indefiniteness of the backward Euler system matrix $\mathbf{K}^\textrm{BE}(\vec{x}^{n+1}_{k})$. Specifically, there are three factors that will conspire to yield indefiniteness in the  backward Euler system matrix: time step size ($\Delta{t}$), nodal mass ($M_{ii}$) and relative magnitude of elasticity parameters ($\mu$,$\lambda$). It may be tempting to adjust these parameters to provide a definite backward Euler system matrix. For example larger nodal mass, smaller $\Delta{t}$ and smaller $\mu,\lambda$ have a better chance of yielding a definite matrix. However, altering the nodal mass or elasticity coefficients would be equivalent to altering the behavior of the material inherent in the governing equations and should therefore be considered off limits. Secondly, when interactivity is desired the time step cannot be decreased arbitrarily. Furthermore, it is important to note that the nodal mass is proportionate to the volume associated with each node. Therefore, as we increase the discrete spatial resolution of our domain, the nodal mass decreases thereby increasing the likelihood of encountering an indefinite backward Euler system matrix as it would behave more and more like the indefinite $\mathbf{K}(\vec{x}^{n+1}_{k})$. Therefore, we see that when maximum performance and high resolution are desired, indefiniteness in the backward Euler system matrix is a distinct possibility. Fortunately, the definiteness fix outline above for $\mathbf{K}(\vec{x}^{n+1}_{k})$ suffices to guarantee definiteness of the backward Euler system matrix.

