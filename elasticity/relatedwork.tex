


% skinning
%
%Skeleton driven skin deformation was first introduced by \cite{Magnenat-Thalmann89}. Since then such techniques have been used extensively, especially the ``linear blend skinning'' technique (aka.\ ``skeleton subspace deformation'' (SSD) or ``enveloping''). However, the limitations of such techniques are well-known and have been the topic of numerous papers \cite{Wang02,Merry06,Kavan08}. Despite improvements, skinning remains, for the most part purely kinematic. It has proven very difficult to get more accurate, physically based deformations (e.g., from self-collisions and contact with other objects). Rather, such phenomena are typically created through a variety of example based approaches \cite{Lewis00,Sloan01}. Although very fast computationally, example based methods often require extreme amounts of user input, especially for contact and collision.

%% elasticity based approaches
%Simulation recently enabled significant advances to character realism in \cite{Irving:2008:SDF} and \cite{clutterbuck:2010:avatar}, albeit with the luxury of extreme computation time. Nevertheless these approaches demonstrated the promise of simulation. Many techniques reduce the accuracy of the elasticity model to help improve performance and interactivity. \cite{Waters90,Chadwick89} first demonstrated the effectiveness of comparatively simple mass/spring based approaches. \cite{Sueda:2008} add interesting anatomic detail using the tendons and bones in the hand, but use simple surface-based skin. \cite{Kry02} use principle component analysis of off-line elasticity simulation to provide interactive physically based SSD. \cite{capell:2005:pb,Capell:2002:ISD:566570.566622,Galopo07} used a skeleton based local rotational model of simple linear elasticity. \cite{Muller:2005:MDB} introduced shape matching, a technique that uses quadratic modal elements defined per lattice cell, allowing realtime albeit less accurate deformations. \cite{Rivers:2007:FFL} extended the accuracy of this
%method while maintaining high performance with a fast SVD. Warped stiffness approaches \cite{Muller:2002:SRD,Muller:2004:IVM} are a more general example of the techniques developed by Cappel et al.\ and use an inexact force differential to yield easily solvable symmetric positive definite (SPD) linearizations. However, \cite{Chao:2010:SGM} recently demonstrated the importance of a more accurate approximation to rotational force differentials lacking in warped stiffness approaches. We  illustrate this important robustness limitation of the warped stiffness approximation in figure~\ref{fig:warpedstiff}. The instability of the method effectively precludes its use in skinning applications. Unfortunately, the more accurate linearizations  yield indefinite systems and thus require more expensive linear algebra techniques (e.g., GMRES). However we demonstrate a solution procedure in the present work that rivals the efficiency available to warped stiffness, but without the robustness difficulties.

% multigrid

Typically elastic simulation requires the solution of large sparse systems. Conjugate gradients is a popular method for solving such systems by virtue of simplicity and low-memory overhead; however, it is plagued by slow convergence (especially with high resolution models). Multigrid techniques potentially avoid these convergence issues, but can be costly to derive for problems over irregular domains. \cite{Zhu:2010:EMM} developed a multigrid approach that achieves nearly optimal convergence properties for incompressible materials in irregular domains. However, their technique for corotational elasticity uses a pseudo-Newton iteration that does not guarantee convergence on the large deformations typical in skeleton driven animation. \cite{Dick:2011:CUDAFEM,Georgii06,Wu04} also examine multigrid methods for rapidly solving the equations of corotational elasticity. However these techniques are based on the warped stiffness approximation to corotational force differentials and demonstrate similar convergence issues as Zhu et al. Multigrid has also been shown to provide excellent parallel performance (e.g., on the GPU in \cite{Dick:2011:CUDAFEM} and on the CPU in \cite{Zhu:2010:EMM}). Our multigrid approach is the first to robustly provide near-interactive performance for non-linear elasticity models with hundreds of thousands of degrees of freedom.
