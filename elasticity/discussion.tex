%##    * Shape richness improving artistic experience
%    * Speed as a game changer
%    * GPU versus CPU
%    * Future work : Physical parameters can still be unintuitive for artists.
%    * Future work : Art direction.

Originally, we were motivated by the desire to make true physically based
elasticity practical for production character rigging. We found that beyond our
expectations, artists were impressed with the shapes a physical deformer could
provide with little manual effort. What usually took days of weight
painting or pose example sculpting was now easy to achieve, and what was
impossible, collision and contact, was now possible.

A major decision in our project was to focus on optimizing for the CPU,
because we were familiar with CPU optimization and because
simulations often need to run on clusters without GPUs.  At the same
time we recognize GPUs are becoming more important because of their power and we
sought to experiment with the GPU. Though we have little experience with the
GPU, we believe our method will perform even better on the GPU with further
optimizations. Further, we note that most GPU experiments tend to compare
against relatively unoptimized CPU implementations, whereas our CPU
implementation was heavily optimized. Regardless we suspect that the GPU
implementation can still be improved significantly. 

Obviously, speed has been a major hurdle for the use of physical simulation in
production, so the fact that our simulations can run at near-interactive rates,
changes the game for artists. Even so, there is plenty of future work to do. One
area is making the parameters (like the Young's modulus) more intuitive for
artists to control, and another is allowing art direction in other ways, such as optimized
control (which also requires fast solvers). \newtext{Improvements to
the initial guess would improve robustness. Especially in the presence
of collisions. It would also allow more accurate self collision treatment in highly
deformed regions.  Finally, our solver
does not support near incompressible materials, and we would be
interested in exploring additional features such as material anisotropy.} But even without this future work, we
believe that our contribution will help create the next generation of appealing
characters. 

% In conclusion, we introduced a new stable corotated linear discretization that
% is robust, requires only one polar decomposition per cell, handles contact
% collision and constraints. We demonstrated a new polar decomposition algorithm
% that requires no trigonometric functions and is 40x faster than any other known
% implementation.  We implemented our new method on both the CPU and GPU and
% demonstrated its efficacy on several production quality examples.
